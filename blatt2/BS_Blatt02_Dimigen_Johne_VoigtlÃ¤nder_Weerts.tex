\documentclass[12pt,a4paper,oneside,ngerman]{article}

\usepackage[left=3cm,right=2.5cm,top=2.5cm]{geometry} % Groesse der Seitenraender definieren
\usepackage[utf8]{inputenc}
%\usepackage [latin1]{inputenc}
\usepackage{listings}
\lstset{breaklines=true, basicstyle=\ttfamily}
\usepackage[official]{eurosym}
\usepackage{multicol}
\usepackage{amsmath}
\usepackage{svg-extract}

\newcommand{\secon}{\operatorname{second}}
\newcommand{\sq}{\operatorname{sqr}}
\newcommand{\head}{\operatorname{head}}
\newcommand{\ones}{\operatorname{ones}}
\newcommand{\oness}{\operatorname{ones}}
\newcommand{\map}{\operatorname{map}}
\newcommand{\inc}{\operatorname{inc}}


\title{Betriebssysteme - Blatt 11}
\author{Steffen Weerts}
\date{Oktober 2022}


% Titel in Kopfzeilen
\usepackage{fancyhdr}
\pagestyle{fancy}
\setlength{\headheight}{20pt}

% % % % % % % % % % % % % % % % % % % % % % % % % % % % % % 
%Variablen
% % % % % % % % % % % % % % % % % % % % % % % % % % % % % % 
\newcommand{\fach}{Betriebssysteme}
\newcommand{\dokumentenTitel}{Abgabe Blatt 1}
\newcommand{\tutor}{Anton Prediger}
\newcommand{\memberOne}{Jakob Dimigen}
\newcommand{\memberTwo}{Sanne Johne}
\newcommand{\memberThree}{Felix Voigtl\"ander}
\newcommand{\memberFour}{Steffen Weerts}
%\newcommand{\group} {Gruppe D}
% % % % % % % % % % % % % % % % % % % % % % % % % % % % % 

% Kopfzeile auf jeder Seite:
\fancyhead[R]{\dokumentenTitel} % Dokument-Titel
\fancyhead[L]{\memberOne, \memberTwo, \memberThree, \memberFour} % Autorennamen


\begin{document}\thispagestyle{plain}

\begin{multicols}{2} % Beginnt zweispaltigen Text fuer Header auf erster Seite
		\hspace{-1cm} % Linken Header-Teil 1cm nach links schieben.
		% Tabelle fuer linke Seite vom Header der ersten Seite
		\begin{tabular}{ll} % Mit l werden die Eintraege linksbuendig
			%Gruppe: & \group \\ 
			Autoren: & \memberOne \\ % Zwischen jeder Spalte ein & einfuegen
			& \memberTwo \\
			& \memberThree  \\
			& \memberFour \\ % beendet eine Tabellenzeile 
			 
		\end{tabular}
		
		\columnbreak % Nun beginnt die rechte Seite des Headers
		\hspace{-1cm} % Rechten Header-Teil 1cm nach links schieben.
		% Tabelle fuer rechte Seite vom Header der ersten Seite
		\raggedleft \begin{tabular}{ll} % p{1cm} bewirkt, dass die rechte Spalte 6cm breit ist.
			Punkte: &  
			%Mit diesem Befehl wird die Zeilenhoehe der folgenden Tabelle um 20% erhoeht.   
			\renewcommand{\arraystretch}{1.2} 
			% Nun kommt eine innere Tabelle in der aeusseren Tabelle, mit der eine Punktetabelle fuer den Tutor erstellt wird:  
			\begin{tabular}{|p{0.8cm}|p{0.8cm}|p{0.8cm}|p{0.8cm}|p{0.8cm}|}
				\hline A3 & A4 & A5 & $\sum\limits^{ }$ \\ \hline
				& & & \\ \hline    
			\end{tabular} \\ Tutor: &  \ \tutor \\
		\end{tabular}
		
	\end{multicols} % Beendet zweispaltigen Text
	
	\begin{center}
		\Large{\fach} \\
		\LARGE{\dokumentenTitel} \\
    \end{center}


\section*{Aufgabe 3}
\begin{enumerate}
	\item[(a)] Das sind zwei unterschiedliche Speicheradressen, da die Pointer auf verschiedene Daten zeigen. Insbesondere liegt davon eine Adresse im Heap, die Andere im Stack.
	
	\item[(b)] Bei mehrfachem Ausführen des Programms ändern sich die Speicheradressen von Ausführung zu Ausführung, wobei die Adresse des Stacks immer auf "4" endet und die Adresse des Heaps immer auf "2a0".
	
	\item [(c)] Bei mehrfachem Ausführen des Programms bleiben die ausgegebenen Speicheradressen gleich
	
	\item[(d)] Laut der zweiten Ausführung des Programms ist an der übergebenen Speicheradresse der Wert "0" gespeichert, obwohl von der ersten Ausführung an diese Adresse der Wert "42" geschrieben wurde.
\end{enumerate}

\section*{Aufgabe 4}
\begin{center}
	\begin{tabular}{ |c|c|c|c|c|c|c|c|c|c|c|c|c|c|c|c|c|c| }
		\hline
		Befehl: & & 31 & 32 & 33 & 11 & 12 & 13 & 21 & 22 & 14 & 34 & 41 & 42 & 23 & 35 & 36 & 43 \\
		\hline
		P1: & N & R & R & A & A & A & B & R & A & N & N & N & N & N & N & N & N \\  
		P2: & N & N & N & N & R & R & A & A & B & B & B & R & A & N & N & N & N \\
		P3: & A & A & A & B & B & B & B & R & R & A & B & R & R & A & A & N & N \\
		P4: & N & N & R & R & R & R & R & R & R & R & A & A & B & B & R & A & N \\
		\hline
	\end{tabular}
\end{center}

\section*{Aufgabe 5}
\begin{enumerate}
	\item[(a)] $A<B$
	\item[(b)] $\neg A\wedge\neg B$.
	\item[(c)] $A<B \vee B<A$.
	\item[(d)] $A<B \vee B<A$.
	\item[(e)] $A<C \vee B<C$.
	\item[(f)] $A<B \vee B<A$.
\end{enumerate}

\end{document}
