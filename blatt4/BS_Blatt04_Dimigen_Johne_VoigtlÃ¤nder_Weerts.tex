\documentclass[12pt,a4paper,oneside,ngerman]{article}

\usepackage[left=3cm,right=2.5cm,top=2.5cm]{geometry} % Groesse der Seitenraender definieren
\usepackage[utf8]{inputenc}
%\usepackage [latin1]{inputenc}
\usepackage{listings}
\lstset{breaklines=true, basicstyle=\ttfamily}
\usepackage[official]{eurosym}
\usepackage{multicol}
\usepackage{amsmath}
\usepackage{svg-extract}

\usepackage{pgfgantt}

\newcommand{\secon}{\operatorname{second}}
\newcommand{\sq}{\operatorname{sqr}}
\newcommand{\head}{\operatorname{head}}
\newcommand{\ones}{\operatorname{ones}}
\newcommand{\oness}{\operatorname{ones}}
\newcommand{\map}{\operatorname{map}}
\newcommand{\inc}{\operatorname{inc}}


\title{Betriebssysteme - Blatt 4}
\author{Steffen Weerts}
\date{Dezember 2022}


% Titel in Kopfzeilen
\usepackage{fancyhdr}
\pagestyle{fancy}
\setlength{\headheight}{20pt}

% % % % % % % % % % % % % % % % % % % % % % % % % % % % % % 
%Variablen
% % % % % % % % % % % % % % % % % % % % % % % % % % % % % % 
\newcommand{\fach}{Betriebssysteme}
\newcommand{\dokumentenTitel}{Abgabe Blatt 4}
\newcommand{\tutor}{Anton Prediger}
\newcommand{\memberOne}{Jakob Dimigen}
\newcommand{\memberTwo}{Sanne Johne}
\newcommand{\memberThree}{Felix Voigtl\"ander}
\newcommand{\memberFour}{Steffen Weerts}
%\newcommand{\group} {Gruppe D}
% % % % % % % % % % % % % % % % % % % % % % % % % % % % % 

% Kopfzeile auf jeder Seite:
\fancyhead[R]{\dokumentenTitel} % Dokument-Titel
\fancyhead[L]{\memberOne, \memberTwo, \memberThree, \memberFour} % Autorennamen


\begin{document}\thispagestyle{plain}

\begin{multicols}{2} % Beginnt zweispaltigen Text fuer Header auf erster Seite
		\hspace{-1cm} % Linken Header-Teil 1cm nach links schieben.
		% Tabelle fuer linke Seite vom Header der ersten Seite
		\begin{tabular}{ll} % Mit l werden die Eintraege linksbuendig
			%Gruppe: & \group \\ 
			Autoren: & \memberOne \\ % Zwischen jeder Spalte ein & einfuegen
			& \memberTwo \\
			& \memberThree  \\
			& \memberFour \\ % beendet eine Tabellenzeile 
			 
		\end{tabular}
		
		\columnbreak % Nun beginnt die rechte Seite des Headers
		\hspace{-1cm} % Rechten Header-Teil 1cm nach links schieben.
		% Tabelle fuer rechte Seite vom Header der ersten Seite
		\raggedleft \begin{tabular}{ll} % p{1cm} bewirkt, dass die rechte Spalte 6cm breit ist.
			Punkte: &  
			%Mit diesem Befehl wird die Zeilenhoehe der folgenden Tabelle um 20% erhoeht.   
			\renewcommand{\arraystretch}{1.2} 
			% Nun kommt eine innere Tabelle in der aeusseren Tabelle, mit der eine Punktetabelle fuer den Tutor erstellt wird:  
			\begin{tabular}{|p{0.8cm}|p{0.8cm}|p{0.8cm}|p{0.8cm}|p{0.8cm}|}
				\hline A7 & $\sum\limits^{ }$ \\ \hline
				& \\ \hline    
			\end{tabular} \\ Tutor: &  \ \tutor \\
		\end{tabular}
		
	\end{multicols} % Beendet zweispaltigen Text
	
	\begin{center}
		\Large{\fach} \\
		\LARGE{\dokumentenTitel} \\
    \end{center}


\section*{Aufgabe 7}
\begin{enumerate}
	\item[(a)] Hier der Ablaufplan nach der Strategie LCFS-PR: \\ \\
	\begin{ganttchart}[hgrid, vgrid, y unit title=0.6cm,title height=1,title label font=\tiny,y unit chart=0.6cm,bar top shift=0.1, bar height=0.8]{1}{23}
		\gantttitlelist{0,10,20,...,220}{1} \\
		\ganttbar[bar right shift=0.5]{P1}{1}{2} \ganttbar[bar left shift=-0.5]{}{12}{14} \\
		\ganttbar[bar left shift=0.5]{P2}{3}{3} \ganttbar[bar right shift=0.5]{}{15}{18} \\
		\ganttbar{P3}{4}{5} \ganttbar[bar left shift=0.5, bar right shift=-0.5]{}{19}{21}\\
		\ganttbar[bar right shift=-0.5]{P4}{6}{6} \ganttbar[bar left shift=-0.5]{}{22}{23} \\
		\ganttbar[bar left shift=-0.5,bar right shift=-0.5]{P5}{7}{11}
	\end{ganttchart}

	Hier der Ablaufplan nach der Strategie RR: \\\\
	\begin{ganttchart}[hgrid, vgrid, y unit title=0.6cm,title height = 1,title label font=\tiny,y unit chart=0.6cm,bar top shift=0.1, bar height=0.8]{1}{23}
		\gantttitlelist{0,10,20,...,220}{1} \\
		\ganttbar{P1}{1}{2} \ganttbar{}{3}{4} \ganttbar{}{9}{10} \\
		\ganttbar{P2}{5}{6} \ganttbar{}{15}{16} \ganttbar{}{22}{22} \\
		\ganttbar{P3}{7}{8} \ganttbar{}{17}{18} \\
		\ganttbar{P4}{11}{12} \ganttbar{}{19}{19} \\
		\ganttbar{P5}{13}{14} \ganttbar{}{20}{21} \ganttbar{}{23}{23}
	\end{ganttchart}

	Hier der Ablaufplan nach der Strategie Multilevel Feedback: \\\\
	\begin{ganttchart}[hgrid, vgrid, y unit title=0.6cm,title height = 1,title label font=\tiny,y unit chart=0.6cm,bar top shift=0.1, bar height=0.8]{1}{23}
		\gantttitlelist{0,10,20,...,220}{1} \\
		\ganttbar{P1}{1}{1}\ganttbar{}{2}{3} \ganttbar{}{16}{18} \\
		\ganttbar{P2}{4}{4} \ganttbar{}{8}{9} \ganttbar{}{19}{20} \\
		\ganttbar{P3}{5}{5} \ganttbar{}{10}{11} \ganttbar{}{21}{21} \\
		\ganttbar{P4}{6}{6} \ganttbar{}{12}{13} \\
		\ganttbar{P5}{7}{7} \ganttbar{}{14}{15} \ganttbar{}{22}{23}
	\end{ganttchart}
	\newpage
	\item[(b)] Hier die Tabelle :) \\
	\begin{tabular}{|l|c|c|c|c|c|c||c|c|}
		\hline
		Strategie & \multicolumn{2}{c|}{LCFS-PR} & \multicolumn{2}{c|}{RR} & \multicolumn{2}{c||}{MLF} & \multicolumn{2}{c|}{Mittelwert} \\
		\hline
		Prozess & $P_1$ & $P_4$ & $P_1$ & $P_4$ & $P_1$ & $P_4$ & $P_1$ & $P_4$ \\
		\hline
		Bedienzeit & 60 & 30 &  60 & 30 &  60 & 30 & 60 & 30 \\
		Antwortzeit & 140 & 160 & 100 & 165 & 180 & 105 & 140 & 143.33 \\
		Wartezeit & 80 & 130 & 40 & 135 & 120 & 75 & 80 & 113.33 \\
		Norm. Antwortzeit & & & & & & & & \\
		\hline
	\end{tabular}
\end{enumerate}

\end{document}
