\documentclass[12pt,a4paper,oneside,ngerman]{article}

\usepackage[left=3cm,right=2.5cm,top=2.5cm]{geometry} % Groesse der Seitenraender definieren
%\usepackage[utf8]{inputenc}
\usepackage [latin1]{inputenc}
\usepackage{listings}
\lstset{breaklines=true, basicstyle=\ttfamily}
\usepackage[official]{eurosym}
\usepackage{multicol}
\usepackage{amsmath}
\usepackage{svg-extract}

\newcommand{\secon}{\operatorname{second}}
\newcommand{\sq}{\operatorname{sqr}}
\newcommand{\head}{\operatorname{head}}
\newcommand{\ones}{\operatorname{ones}}
\newcommand{\oness}{\operatorname{ones}}
\newcommand{\map}{\operatorname{map}}
\newcommand{\inc}{\operatorname{inc}}


\title{Betriebssysteme - Blatt 11}
\author{Steffen Weerts}
\date{Oktober 2022}


% Titel in Kopfzeilen
\usepackage{fancyhdr}
\pagestyle{fancy}
\setlength{\headheight}{20pt}

% % % % % % % % % % % % % % % % % % % % % % % % % % % % % % 
%Variablen
% % % % % % % % % % % % % % % % % % % % % % % % % % % % % % 
\newcommand{\fach}{Betriebssysteme}
\newcommand{\dokumentenTitel}{Abgabe Blatt 1}
\newcommand{\tutor}{Anton Prediger}
\newcommand{\memberOne}{Jakob Dimigen}
\newcommand{\memberTwo}{Sanne Johne}
\newcommand{\memberThree}{Felix Voigtl\"ander}
\newcommand{\memberFour}{Steffen Weerts}
%\newcommand{\group} {Gruppe D}
% % % % % % % % % % % % % % % % % % % % % % % % % % % % % 

% Kopfzeile auf jeder Seite:
\fancyhead[R]{\dokumentenTitel} % Dokument-Titel
\fancyhead[L]{\memberOne, \memberTwo, \memberThree, \memberFour} % Autorennamen


\begin{document}\thispagestyle{plain}

\begin{multicols}{2} % Beginnt zweispaltigen Text fuer Header auf erster Seite
		\hspace{-1cm} % Linken Header-Teil 1cm nach links schieben.
		% Tabelle fuer linke Seite vom Header der ersten Seite
		\begin{tabular}{ll} % Mit l werden die Eintraege linksbuendig
			%Gruppe: & \group \\ 
			Autoren: & \memberOne \\ % Zwischen jeder Spalte ein & einfuegen
			& \memberTwo \\
			& \memberThree  \\
			& \memberFour \\ % beendet eine Tabellenzeile 
			 
		\end{tabular}
		
		\columnbreak % Nun beginnt die rechte Seite des Headers
		\hspace{-1cm} % Rechten Header-Teil 1cm nach links schieben.
		% Tabelle fuer rechte Seite vom Header der ersten Seite
		\raggedleft \begin{tabular}{ll} % p{1cm} bewirkt, dass die rechte Spalte 6cm breit ist.
			Punkte: &  
			%Mit diesem Befehl wird die Zeilenhoehe der folgenden Tabelle um 20% erhoeht.   
			\renewcommand{\arraystretch}{1.2} 
			% Nun kommt eine innere Tabelle in der aeusseren Tabelle, mit der eine Punktetabelle fuer den Tutor erstellt wird:  
			\begin{tabular}{|p{0.8cm}|p{0.8cm}|p{0.8cm}|p{0.8cm}|p{0.8cm}|}
				\hline A1 & A2 & $\sum\limits^{ }$ \\ \hline
				& & \\ \hline    
			\end{tabular} \\ Tutor: &  \ \tutor \\
		\end{tabular}
		
	\end{multicols} % Beendet zweispaltigen Text
	
	\begin{center}
		\Large{\fach} \\
		\LARGE{\dokumentenTitel} \\
    \end{center}


\section*{Aufgabe 1}
\begin{enumerate}
	\item[(a)]
	\lstinputlisting[language=C]{aufgabe1_dynamische_speicherverwaltung_in_c/monitoring_alloc.c}
	
	\item[(b)]
	Das Beispielprogramm erstellt in der \textit{leaking\_function} Speicherbl\"ocke und gibt lediglich den letzten Spiecherblock zur\"uck, der dann freigegeben wird.
	Alle vorherigen Speicherbl\"ocke bleiben erhalten und auf diese kann nicht mehr zugegriffen werden.
\end{enumerate}

\section*{Aufgabe 2}
\begin{enumerate}
	\item[tree.c]
	\lstinputlisting[language=C]{aufgabe2_binaerer_baum_in_c/tree.c}
	
	\item[main.c]
	\lstinputlisting[language=C]{aufgabe2_binaerer_baum_in_c/main.c}
\end{enumerate}



\end{document}
